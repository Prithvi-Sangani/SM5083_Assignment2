\documentclass[journal,12pt,twocolumn]{IEEEtran}

\usepackage{setspace}
\usepackage{gensymb}
\singlespacing
\usepackage[cmex10]{amsmath}

\usepackage{amsthm}
\usepackage{amsmath} 

\usepackage{mathrsfs}
\usepackage{txfonts}
\usepackage{stfloats}
\usepackage{bm}
\usepackage{cite}
\usepackage{cases}
\usepackage{subfig}

\usepackage{longtable}
\usepackage{multirow}

\usepackage{enumitem}
\usepackage{mathtools}
\usepackage{steinmetz}
\usepackage{tikz}
\usepackage{circuitikz}
\usepackage{verbatim}
\usepackage{tfrupee}
\usepackage[breaklinks=true]{hyperref}
\usepackage{graphicx}
\usepackage{tkz-euclide}

\usetikzlibrary{calc,math}
\usepackage{listings}
    \usepackage{color}                                            %%
    \usepackage{array}                                            %%
    \usepackage{longtable}                                        %%
    \usepackage{calc}                                             %%
    \usepackage{multirow}                                         %%
    \usepackage{hhline}                                           %%
    \usepackage{ifthen}                                           %%
    \usepackage{lscape}     
\usepackage{multicol}
\usepackage{chngcntr}

\DeclareMathOperator*{\Res}{Res}

\renewcommand\thesection{\arabic{section}}
\renewcommand\thesubsection{\thesection.\arabic{subsection}}
\renewcommand\thesubsubsection{\thesubsection.\arabic{subsubsection}}

\renewcommand\thesectiondis{\arabic{section}}
\renewcommand\thesubsectiondis{\thesectiondis.\arabic{subsection}}
\renewcommand\thesubsubsectiondis{\thesubsectiondis.\arabic{subsubsection}}


\hyphenation{op-tical net-works semi-conduc-tor}
\def\inputGnumericTable{}                                 %%

\lstset{
%language=C,
frame=single, 
breaklines=true,
columns=fullflexible
}
\begin{document}


\newtheorem{theorem}{Theorem}[section]
\newtheorem{problem}{Problem}
\newtheorem{proposition}{Proposition}[section]
\newtheorem{lemma}{Lemma}[section]
\newtheorem{corollary}[theorem]{Corollary}
\newtheorem{example}{Example}[section]
\newtheorem{definition}[problem]{Definition}

\newcommand{\BEQA}{\begin{eqnarray}}
\newcommand{\EEQA}{\end{eqnarray}}
\newcommand{\define}{\stackrel{\triangle}{=}}
\bibliographystyle{IEEEtran}
\raggedbottom
\setlength{\parindent}{0pt}
\providecommand{\mbf}{\mathbf}
\providecommand{\pr}[1]{\ensuremath{\Pr\left(#1\right)}}
\providecommand{\qfunc}[1]{\ensuremath{Q\left(#1\right)}}
\providecommand{\sbrak}[1]{\ensuremath{{}\left[#1\right]}}
\providecommand{\lsbrak}[1]{\ensuremath{{}\left[#1\right.}}
\providecommand{\rsbrak}[1]{\ensuremath{{}\left.#1\right]}}
\providecommand{\brak}[1]{\ensuremath{\left(#1\right)}}
\providecommand{\lbrak}[1]{\ensuremath{\left(#1\right.}}
\providecommand{\rbrak}[1]{\ensuremath{\left.#1\right)}}
\providecommand{\cbrak}[1]{\ensuremath{\left\{#1\right\}}}
\providecommand{\lcbrak}[1]{\ensuremath{\left\{#1\right.}}
\providecommand{\rcbrak}[1]{\ensuremath{\left.#1\right\}}}
\theoremstyle{remark}
\newtheorem{rem}{Remark}
\newcommand{\sgn}{\mathop{\mathrm{sgn}}}
\providecommand{\abs}[1]{\left\vert#1\right\vert}
\providecommand{\res}[1]{\Res\displaylimits_{#1}} 
\providecommand{\norm}[1]{\left\lVert#1\right\rVert}
%\providecommand{\norm}[1]{\lVert#1\rVert}
\providecommand{\mtx}[1]{\mathbf{#1}}
\providecommand{\mean}[1]{E\left[ #1 \right]}
\providecommand{\fourier}{\overset{\mathcal{F}}{ \rightleftharpoons}}
%\providecommand{\hilbert}{\overset{\mathcal{H}}{ \rightleftharpoons}}
\providecommand{\system}{\overset{\mathcal{H}}{ \longleftrightarrow}}
	%\newcommand{\solution}[2]{\textbf{Solution:}{#1}}
\newcommand{\solution}{\noindent \textbf{Solution: }}
\newcommand{\cosec}{\,\text{cosec}\,}
\providecommand{\dec}[2]{\ensuremath{\overset{#1}{\underset{#2}{\gtrless}}}}
\newcommand{\myvec}[1]{\ensuremath{\begin{pmatrix}#1\end{pmatrix}}}
\newcommand{\mydet}[1]{\ensuremath{\begin{vmatrix}#1\end{vmatrix}}}
\numberwithin{equation}{subsection}

\makeatletter
\@addtoreset{figure}{problem}
\makeatother
\let\StandardTheFigure\thefigure
\let\vec\mathbf

\renewcommand{\thefigure}{\theproblem}

\def\putbox#1#2#3{\makebox[0in][l]{\makebox[#1][l]{}\raisebox{\baselineskip}[0in][0in]{\raisebox{#2}[0in][0in]{#3}}}}
     \def\rightbox#1{\makebox[0in][r]{#1}}
     \def\centbox#1{\makebox[0in]{#1}}
     \def\topbox#1{\raisebox{-\baselineskip}[0in][0in]{#1}}
     \def\midbox#1{\raisebox{-0.5\baselineskip}[0in][0in]{#1}}
\vspace{3cm}
\title{Assignment 2}
\author{S Prithvi \\ CE20RESCH13001}
\maketitle
\newpage
\bigskip
\renewcommand{\thefigure}{\theenumi}
\renewcommand{\thetable}{\theenumi}
\section{Chapter III, Example III, Q.2}
In what ratio is the join of \myvec{2\\3} and \myvec{3\\4} is divided by the join of \myvec{3\\3} and \myvec{5\\-2}
\subsection{Solution}
Let the given points be
\begin{equation}
\vec{A} = \ \myvec{2 \\3} \ ; \ \vec{B} \ = \myvec{3\\ 4}; \ 
\vec{C} = \ \myvec{3 \\3} \ ; \ \vec{D} \ = \myvec{5\\ -2} \label{1}
\end{equation}
Let $\vec{X}$ be the point of intersection of lines joining $\vec{A}$,$\vec{B}$ \& $\vec{C}$,$\vec{D}$. From the definition of the slopes,
\begin{equation}
\vec{B} - \vec{X} = k_{1}(\vec{B}-\vec{A}) \label{2} 
\end{equation}
\begin{equation}
    \vec{D} - \vec{X} = k_2(\vec{D} - \vec{C}) \label{3}
\end{equation}
Subtracting \eqref{3} from \eqref{2}, we get
\begin{equation}
\vec{B} - \vec{D} = k_1(\vec{B} - \vec{A})-k_2(\vec{D} - \vec{C}) \label{4}    
\end{equation}
Substituting \eqref{1} in \eqref{4}
\begin{equation}
 \myvec{-2\\ 6} = k_1\myvec{1 \\ 1}- k_2\myvec{2\\ -5} \label{5}
\end{equation}
\begin{equation}
\myvec{1 & -2 \\ 1 & 5} \myvec{k_1\\k_2} = \myvec{-2\\6} \label{6}    
\end{equation}
\begin{equation}
\myvec{k_1\\k_2} = \myvec{1 & -2 \\ 1 & 5}^{-1}\myvec{-2\\6}
\end{equation}
\begin{equation}
\myvec{k_1\\k_2} = \myvec{\frac{2}{7}\\ \frac{8}{7}} \label{8}    
\end{equation}
Substituting \eqref{8} in \eqref{2},
\begin{equation}
\myvec{3\\4} - \vec{X} = \frac{2}{7}\myvec{3\\4} -\frac{2}{7}\myvec{2\\3} \label{9} 
\end{equation}
\begin{equation}
\vec{X} = \myvec{\frac{19}{7}\\\frac{26}{7}} \label{10}    
\end{equation}
ratio,r by which the intersection divides the line joining points $\vec{A}$ \& $\vec{B}$ is given as
\begin{equation}
r = \frac{\norm{\vec{B} -\vec{X}}}{\norm{\vec{A} -\vec{X}}} \label{11}
\end{equation}
\begin{equation}
r = \frac{\norm{\myvec{3\\4} -\myvec{\frac{19}{7}\\ \frac{26}{7}}}}{\norm{\myvec{2\\3} -\myvec{\frac{19}{7}\\\frac{26}{7}}}} \label{12}   
\end{equation}
\begin{equation}
r = \frac{2}{5} \label{13}   
\end{equation}

\end{document}









In order to convert to rectangular coordinate system, the y-axis should be rotated by $30^{\circ}$ in anti-clockwise.
Transformed coordinates of \myvec{x_1\\y_1} \& \myvec{x_2\\y_2} be \myvec{x_3\\y_3} \& \myvec{x_4\\y_4} respectively.

$x_3$ = O$X_1$ + $X_1$$X_3$ = $x_1$+$y_1$$\cos{60^{\circ}$\\
$y_3$ = O$Y_1$$\cos{30^{\circ}$$ = $y_1$$\cos{30^{\circ}$\\
\begin{equation}
\myvec{x_3\\y_3} = \myvec{1 \ \cos{60^{\circ}} \\ 0 \  \cos{30^{\circ}}} \myvec{x_1\\y_1}\label{eq:1.0.2}   
\end{equation}
Similarly,\\
$x_4$ = O$X_2$ + $X_2$$X_4$ = $x_2$+$y_2$\cos{60^{\circ}$$ \\
$y_4$ = O$Y_2$$\cos{30^{\circ}$$  =\ $y_2$$\cos{30^{\circ}$\\
\begin{equation}
\myvec{x_4\\y_4} = \myvec{1 \ \cos{60^{\circ}} \\ 0 \  \cos{30^{\circ}}} \myvec{x_2\\y_2}\label{eq:1.0.3}
\end{equation}
The generalised equation for transformed coordinates $\myvec{x_t\\y_t}$ when the angle between axes `$\theta$ is,
\begin{equation}
\boxed{\myvec{x_t\\y_t} = \myvec{1   \ \  \cos{(\theta)} \\ 0 \ \ \sin{(\theta)}} \myvec{x\\y}} \label{eq:1.1.4}  
\end{equation}
Let the transformed point be $\vec{X_t}$, $\vec{T}$ be the transformation matrix and the point in angular axes be $\vec{X}$, \eqref{eq:1.1.4} can be written as 
\begin{equation}
 \vec{X_t} = \vec{T} \ \vec{X}\label{eq:1.1.5}  
\end{equation}
Substituting \eqref{1.0.1} in \eqref{eq:1.0.2} \& \eqref{eq:1.0.3}
 \begin{equation}
  \myvec{x_3\\y_3} = \myvec{\frac{13}{2}\\\frac{5\sqrt{3}}{2}};\ \myvec{x_4\\y_4} = \myvec{10\\3\sqrt{3}}\label{eq:1.0.5}    
 \end{equation}
The distance between points is a norm of the distance vector,
\begin{equation}
d = \norm{\vec{X_t_1}-\vec{X_t_2}} \label{eq:1.1.7}
\end{equation}
Substituting \eqref{eq:1.1.5} in \eqref{eq:1.1.7},
\begin{equation}
d = \norm{\vec{T}\vec{X_1}-\vec{T}\vec{X_2}} \label{eq:1.1.8}
\end{equation}
\begin{equation}
d = \norm{\vec{T}(\vec{X_1}-\vec{X_2})} \label{eq:1.1.123}
\end{equation}
\begin{equation}
d = (\vec{X_1}-\vec{X_2})^\top \  \vec{T}^\top \  \vec{T} \ (\vec{X_1}-\vec{X_2})  \label{eq:1.1.9}
\end{equation}
\begin{equation}
d = \sqrt{13} \ units  \label{eq:1.1.10}
\end{equation}
Above results shows that the distance remains constant between the points irrespective of coordinate system and by \eqref{1.0.1} \& \eqref{eq:1.0.5} only the position vector of the point changes with the transformation of coordinate system 




\counterwithin{figure}{section}
\begin{figure}[!ht]
	\centering
	\includegraphics[width=\columnwidth]{fig1.png}
	\caption{Points defined on angular \& rectangular axes}
	\end{figure}
 \\ \\ %if not used the figure is going to the middle of the page
\counterwithin{figure}{section}
\begin{figure}[!ht]
    \centering
    \includegraphics[width=\columnwidth]{download.png}
    \caption{Points plotted in Python}
    \label{Fig2: Points plotted in Python}
\end{figure}






